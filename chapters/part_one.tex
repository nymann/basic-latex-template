\section{Code}
\subsection{Python}
\inputminted{python}{code/ttf-to-otp.py}

\subsection{AWK}
The following is a snippet from the book titled The AWK Programming Language.

\subsubsection{Password-File Checking}
The password file on a Unix system contains the name of and other information
about authorized users. Each line of the password file has 7 fields, seperated
by colons:
\small\begin{verbatim}
    root:qyxRi2uhuVjrg:0:2::/:
    bwk:1L./v6iblzzNE:9:1:Brian Kerninghan:/usr/bwk:
    ava:otxs1oTVoyvMQ:15:1:Al Aho:/usr/ava:
    uucp:xutIBs2hKtcls:48:1:uucp daemon:/usr/lib/uucp:uucico
    pjw:xNqy//GDc8FFg:170:2:Peter Winberger:/usr/pjw:
    mark:j0z1fuQmqIvdE:374:1:Mark Kernighan:/usr/bwk/mark:
    ...
\end{verbatim}
\normalsize
The first field is the user's login name, which should be alphanumeric. The
second is an encrypted version of the password; if this field is empty, anyone
can log in pretending to be that user, while if there is a password, only
people who know the password can log in. The third and fourth fields are
supposed to be numeric. The sixth field should begin with /. The following
program prints all lines that fail to satisfy these criteria, along with the
number of the erroneous line and an approritate diagnostic message. Running
this program every night is a small part of keeping a system healthy and safe
from intruders.

\inputminted{awk}{code/password-check.awk}

\section{Italic}
\textit{\lipsum[2]}

\section{Bold}
\textbf{\lipsum[4]}
